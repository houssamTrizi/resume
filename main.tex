%% start of file `template.tex'.
%% Copyright 2006-2013 Xavier Danaux (xdanaux@gmail.com).
%
% This work may be distributed and/or modified under the
% conditions of the LaTeX Project Public License version 1.3c,
% available at http://www.latex-project.org/lppl/.


\documentclass[12pt,a4paper,roman]{moderncv}        % possible options include font size ('10pt', '11pt' and '12pt'), paper size ('a4paper', 'letterpaper', 'a5paper', 'legalpaper', 'executivepaper' and 'landscape') and font family ('sans' and 'roman')

% modern themes
\moderncvstyle{banking}                            % style options are 'casual' (default), 'classic', 'oldstyle' and 'banking'
\moderncvcolor{blue}                                % color options 'blue' (default), 'orange', 'green', 'red', 'purple', 'grey' and 'black'
%\renewcommand{\familydefault}{\sfdefault}         % to set the default font; use '\sfdefault' for the default sans serif font, '\rmdefault' for the default roman one, or any tex font name
\nopagenumbers{}                                  % uncomment to suppress automatic page numbering for CVs longer than one page

% character encoding
\usepackage[utf8]{inputenc}
\usepackage{fontawesome}
\usepackage{fontspec}
\usepackage{tabularx}
\usepackage{ragged2e}
% if you are not using xelatex ou lualatex, replace by the encoding you are using
%\usepackage{CJKutf8}                              % if you need to use CJK to typeset your resume in Chinese, Japanese or Korean

% adjust the page margins
\usepackage[scale=0.85]{geometry}
\usepackage{multicol}
%\setlength{\hintscolumnwidth}{3cm}                % if you want to change the width of the column with the dates
%\setlength{\makecvtitlenamewidth}{10cm}           % for the 'classic' style, if you want to force the width allocated to your name and avoid line breaks. be careful though, the length is normally calculated to avoid any overlap with your personal info; use this at your own typographical risks...

\usepackage{import}

% personal data
\name{Jihane}{SERRAR}
% \title{Curriculum Vitae}                               % optional, remove / comment the line if not wanted
\address{121 avenue du 18 juin 1940, 92500 Rueil Malmaison }{}{}% optional, remove / comment the line if not wanted; the "postcode city" and and "country" arguments can be omitted or provided empty
% \phone[mobile]{909-839-3097}                   % optional, remove / comment the line if not wanted
% \phone[fixed]{01234 123456}                    % optional, remove / comment the line if not wanted
%\phone[fax]{+3~(456)~789~012}                      % optional, remove / comment the line if not wanted
% \email{xpan1@swarthmore.edu}                               % optional, remove / comment the line if not wanted
% \homepage{shawnpan.me}                         % optional, remove / comment the line if not wanted
% \extrainfo{}                 % optional, remove / comment the line if not wanted
\photo[70pt][1.5pt]{images/picture.jpg}                     % optional, remove / comment the line if not wanted; '64pt' is the height the picture must be resized to, 0.4pt is the thickness of the frame around it (put it to 0pt for no frame) and 'picture' is the name of the picture file
%\quote{Some quote}                                 % optional, remove / comment the line if not wanted

% to show numerical labels in the bibliography (default is to show no labels); only useful if you make citations in your resume
%\makeatletter
%\renewcommand*{\bibliographyitemlabel}{\@biblabel{\arabic{enumiv}}}
%\makeatother
%\renewcommand*{\bibliographyitemlabel}{[\arabic{enumiv}]}% CONSIDER REPLACING THE ABOVE BY THIS

% bibliography with mutiple entries
%\usepackage{multibib}
%\newcites{book,misc}{{Books},{Others}}
  
\newcommand*{\customcventry}[7][.25em]{
  \begin{tabular}{@{}l} 
    {\bfseries #4}
  \end{tabular}
  \hfill% move it to the right
  \begin{tabular}{l@{}}
     {\bfseries #5}
  \end{tabular} \\
  \begin{tabular}{@{}l} 
    {\itshape #3}
  \end{tabular}
  \hfill% move it to the right
  \begin{tabular}{l@{}}
     {\itshape #2}
  \end{tabular}
  \ifx&#7&%
  \else{\\%
    \begin{minipage}{\maincolumnwidth}%
      \small#7%
    \end{minipage}}\fi%
  \par\addvspace{#1}}

\newcommand*{\customcvproject}[4][.25em]{
%   \vfill\noindent
  \begin{tabular}{@{}l} 
    {\bfseries #2}
  \end{tabular}
  \hfill% move it to the right
  \begin{tabular}{l@{}}
     {\itshape #3}
  \end{tabular}
  \ifx&#4&%
  \else{\\% 
   \begin{minipage}{\maincolumnwidth}%
      \small#4%
    \end{minipage}}\fi%
  \par\addvspace{#1}}

\setlength{\tabcolsep}{12pt}

%----------------------------------------------------------------------------------
%            content
%----------------------------------------------------------------------------------
\begin{document}
%\begin{CJK*}{UTF8}{gbsn}                          % to typeset your resume in Chinese using CJK
%-----       resume       ---------------------------------------------------------
\makecvtitle
\vspace*{-23mm}

\begin{center}
\begin{tabular}{ c c c c }
% \faGlobe\enspace xing.com/profile/HoussamEddineTrizi%
 \faEnvelopeO\enspace jihane.serrar1@gmail.com  & \faMobile\enspace +33 (0) 75-117-0225
\end{tabular}
\end{center}

\section{EDUCATION}
{\customcventry{september 2019 - Present}{Masters degree - Probability and Finance (ex DEA EL KAROUI)}{Polytechnique}{Paris, France}{}{}}
\vspace*{3mm}
{\customcventry{June 2017}{Engineer's degree - Modeling and Computer Science}{Mohammedia Engineering School}{Rabat, Morocco}{}{}}
\vspace*{3mm}
{\customcventry{June 2012}{Bachelor's Degree major: Mathematics}{Higher School Prepatory Classes}{Tangier, Morocco}{}{}}

\vspace*{4mm}


\section{EXPERIENCE}
%
{\customcventry{Mar 2018 - August 2019}{Pricing and Structuration Engineer}{Societe Generale Corporate and Investment Banking}{Casablanca, Morocco}{}{}
{\begin{itemize}
  \item Price the cost of various forex products and solutions requested by clients (distributors).
  \item Execute those transactions with sales and traders.
    \item Participate in designing and pricing innovative forex structured products.

  
\end{itemize}
}

\vspace*{3mm}

{\customcventry{April 2017 – Oct 2017}{INRIA}{The National Institute for Research in Computer Science Intern}{Lille, France}{}{}
{\begin{itemize}
  \item Model the electric vehicle routing problem (EVRP).
  \item Develop algorithms (heuristics) to solve the EVRP. 
  \end{itemize}
}

\vspace*{3mm}

{\customcventry{July 2016 – Aug 2016}{CEAC}{Electrical Construction And Metering Equipment Intern}{Fez, Morocco}{}{}
{\begin{itemize}
  \item Optimize the production line of metering equipment.
  \item Optimize the use and manage the stock.
  \item Improve the safety and health system at work according to the referential OHSAS: 2007.  
\end{itemize}
}
\vspace*{4mm}

\section{SKILLS}
\begin{minipage}{\maincolumnwidth}%
	\small{

\begin{tabular}{llllllllll}
Ms Office  &  & &  & C++ & & & & & Excel VBA\\
SQL  &  & &  & LaTex & & & & & \\
\end{tabular}
\end{minipage}%
      
}



\vspace*{4mm}

\section{LANGUAGES}
\begin{minipage}{\maincolumnwidth}%
	\small{
    	\begin{itemize}
          \item French : Fluent  
          \item English : Fluent 
          \item Arabic : Native
		\end{itemize}}%
\end{minipage}%
      
}



% Publications from a BibTeX file without multibib
%  for numerical labels: \renewcommand{\bibliographyitemlabel}{\@biblabel{\arabic{enumiv}}}% CONSIDER MERGING WITH PREAMBLE PART
%  to redefine the heading string ("Publications"): \renewcommand{\refname}{Articles}
\nocite{*}
\bibliographystyle{plain}
%\bibliography{publications}                        % 'publications' is the name of a BibTeX file

% Publications from a BibTeX file using the multibib package
%\section{Publications}
%\nocitebook{book1,book2}
%\bibliographystylebook{plain}
%\bibliographybook{publications}                   % 'publications' is the name of a BibTeX file
%\nocitemisc{misc1,misc2,misc3}
%\bibliographystylemisc{plain}
%\bibliographymisc{publications}                   % 'publications' is the name of a BibTeX file

%-----       letter       ---------------------------------------------------------

\end{document}


%% end of file `template.tex'.
